\documentclass[11pt,reqno,letter]{article}

\usepackage{amsmath}
\usepackage{amsfonts}
\usepackage{amsthm}
\usepackage{graphicx}
%\usepackage{epstopdf}
\usepackage{hyperref}
\hypersetup{colorlinks=true}
\usepackage[left=1in,right=1in,top=0.9in,bottom=0.9in]{geometry}
\usepackage{multirow}
\usepackage{verbatim}
\usepackage{fancyhdr}
\usepackage{mdframed}
%\usepackage[small,compact]{titlesec}
\usepackage{listings}

\usepackage{natbib}
\renewcommand{\cite}{\citet}

%\usepackage{pxfonts}
%\usepackage{isomath}
\usepackage{mathpazo}
%\usepackage{arev} %     (Arev/Vera Sans)
%\usepackage{eulervm} %_   (Euler Math)
%\usepackage{fixmath} %  (Computer Modern)
%\usepackage{hvmath} %_   (HV-Math/Helvetica)
%\usepackage{tmmath} %_   (TM-Math/Times)
%\usepackage{cmbright}
%\usepackage{ccfonts} \usepackage[T1]{fontenc}
%\usepackage[garamond]{mathdesign}
\usepackage{color}
\usepackage[normalem]{ulem}

\newtheorem{theorem}{Theorem}[section]
\newtheorem{conjecture}{Conjecture}[section]
\newtheorem{corollary}{Corollary}[section]
\newtheorem{lemma}{Lemma}[section]
\newtheorem{proposition}{Proposition}[section]
\theoremstyle{definition}
\newtheorem{assumption}{}[section]
%\renewcommand{\theassumption}{C\arabic{assumption}}
\newtheorem{definition}{Definition}[section]
\newtheorem{step}{Step}[section]
\newtheorem{remark}{Comment}[section]
\newtheorem{example}{Example}[section]
\newtheorem*{example*}{Example}

\linespread{1.0}

\pagestyle{fancy}
%\renewcommand{\sectionmark}[1]{\markright{#1}{}}
\fancyhead{}
\fancyfoot{}
%\fancyhead[LE,LO]{\tiny{\thepage}}
\fancyhead[CE,CO]{\tiny{\rightmark}}
\fancyfoot[C]{\small{\thepage}}
\renewcommand{\headrulewidth}{0pt}
\renewcommand{\footrulewidth}{0pt}

\newtheoremstyle{problem}% name
{12pt}% Space above
{5pt}% Space below
{}% Body font
{}% Indent amount
{\bfseries}% Theorem head font
{:}% Punctuation after theorem head
{.5em}% Space after theorem head
{}% Theorem head spec (can be left empty, meaning `normal')

\theoremstyle{problem}
\newtheorem{prob}{Problem}
\newenvironment{problem} 
  {\begin{mdframed}\begin{prob}$\,$}
  {\end{prob}\end{mdframed}}


\newtheoremstyle{solution}% name
{2pt}% Space above
{12pt}% Space below
{}% Body font
{}% Indent amount
{\bfseries}% Theorem head font
{:}% Punctuation after theorem head
{.5em}% Space after theorem head
{}% Theorem head spec (can be left empty, meaning `normal')
\newtheorem{soln}{Solution}

\newenvironment{solution}
  {\begin{mdframed}\begin{soln}$\,$}
  {\end{soln}\end{mdframed}}


\lstset{language=R}
\lstset{basicstyle=\footnotesize\ttfamily,breaklines=true}
\lstset{keywordstyle=\color[rgb]{0,0,1},                                        % keywords
        commentstyle=\color[rgb]{0.133,0.545,0.133},    % comments
        stringstyle=\color[rgb]{0.627,0.126,0.941}      % strings
}       
\lstset{
  showstringspaces=false,       % not emphasize spaces in strings 
  %columns=fixed,
  %numbersep=3mm, numbers=left, numberstyle=\tiny,       % number style
  frame=1mm,
  framexleftmargin=6mm, xleftmargin=6mm         % tweak margins
}


\title{Reproducing Grieco \& McDevitt (2017)}
\author{Paul Schrimpf}
\date{Due: January 26, 2017}

\begin{document}
\maketitle

In this problem set, we will attempt to reproduce the main results of
\cite{grieco2017}. 

\begin{problem}
  \url{http://tryr.codeschool.com/} is an interactive introduction to
  R. Please work through it if you have not used R before. If you're
  already familiar with R, then you can skip this.
\end{problem}

If you're new to R, here is some advice about additional tools to
use. R itself comes with either no GUI nor text-editor (on Linux) or a
basic GUI with a limited text editor, (on Windows and I'd guess on
Mac, but I don't know). There are numerous programs that provide a
nicer way of working with R. The most popular is Rstudio. It gives
nice syntax highlighting, easier debugging, etc; it is somewhat
similar to Matlab's GUI. A potentially useful, but not essential, tool
for this assignment is the
\href{http://rmarkdown.rstudio.com/}{rmarkdown package}. It lets you
combine R code and text into a single document and produces nice
looking output in multiple formats. I often use it for preliminary
data work that I'm just looking at or sharing with coauthors and still
making changes frequently. For more complete papers, I prefer to keep
my code and the text of the paper separate. 

\section{Explore the data}

I downloaded the data for this problem set from
\url{https://dialysisdata.org/content/dialysis-facility-report-data}. As
in \cite{grieco2017} the data comes from Dialysis Facility Reports
(DFRs) created under contract to the Centers for Medicare and Medicaid
Services (CMS). However, there are some differences. Most notably,
this data covers 2006-2014, instead of 2004-2008 as in
\cite{grieco2017}. 

We will begin our analysis with some exploratory statistics and
figures. There are at least two reasons for this. First, we want to
check for any anomalies in the data, which may indicate an error in
our code, our understanding of the data, or the data itself. Second,
we should try to see if there are any striking patterns in the data
that deserve extra attention.

The script
\href{https://bitbucket.org/paulschrimpf/econ565/src/master/assignments/production-R/dialysis/downloadDialysisData.R}{downloadDialysisData.R}
downloads, combines, and cleans the data from
\url{https://dialysisdata.org/content/dialysis-facility-report-data}. You
can run this script yourself if you'd like, but it will take some time
and download 825M of data. The result of the script is
\href{http://faculty.arts.ubc.ca/pschrimpf/565/dialysisFacilityReports.Rdata}{dialysisFacilityReports.Rdata},
an R data file containing most of the variables used by
\cite{grieco2017}.

Load the data into R by typing
\begin{lstlisting}
load("dialysisFacilityReport.R")
\end{lstlisting}
You could enter this directly into R's command line, but it is always
better to write your commands in a script (or rmarkdown document) and
run the script. You are bound to make mistakes, and it will be easier
to fix them if you save your commands in a file instead of entering
them one by one.

Your workspace should now contain a data frame named ``dialysis.'' Data
frames are how R stores data. A data frame is basically a matrix where
the columns represent variables and the rows are observations. The
variables/columns have names. To list the names of a data frame, run
\begin{lstlisting}
names(df)
\end{lstlisting}
The meanings of these variables are listed in Table \ref{tab:def}.

\begin{table}\caption{Variable definitions\label{tab:def} }
  \begin{minipage}{\linewidth}
    \begin{tabular}{ll}
    \textbf{Variable} & \textbf{Definition} \\
      provfs & provided identifier \\ 
      year & year of data \\
      city & city of provider \\
      name & provider name \\
      state & state of provider \\
      network & network number$^{*}$ \\
      chain.name & name of chain if provider is part of one \\
      profit.status & whether for profit or non-profit \\
      comorbities & average number of patient comorbidities \\
      hemoglobin & average patient hemoglobin level \\
      std.mortality & standardized mortality ratio \\
      std.hosp.days & standardized hospitalization days \\
      std.hosp.admit & standardized hospitalization admittance rate \\
      pct.septic & percent of patients hospitalized due to septic infection \\
      n.hosp.admt & number of hospitalizations \\
      stations & number of dialysis stations \\
      total.staff & total staff \\
      dieticiansFT & full-time renal dieticians \\
      dieticiansPT & part-time renal dieticians \\ 
      nurseFT & full-time nurses ($>32$ hours/week) \\
      nursePT & part-time nurses ($<32$ hours/week) \\
      ptcareFT & full-time patient care technicians \\
      ptcarePT & part-time patient care technicians \\
      social.workerFT & full-time social workers \\
      social.workerPT & part-time social workers \\
      patient.months & number of patient-months treated during the
                       year \\
      pct.fistula & the percentage of
                    patient months in which the patient received dialysis through arteriovenous (AV)
                    fistulae \\
      pct.female & percent of female patients \\
      patient.age & average age of patients \\
      patient.esrd.years & average number of years patients have had
                           end stage renal disease \\
      treatment.type & types of treatment provided at facility \\
      inspect.date & date of most recent inspection \\
      inspect.result & result of most recent inspection \\
      inspect.cfc.cites & number of condition for coverage
                          deficiencies in most recent inspection \\
      inspect.std.cites & number of standard deficiencies in most
                          recent inspection \\
      days.since.inspection & days since last inspection 
  \end{tabular}
  \footnotetext{$^*$ I am unsure of the meaning of these
    variables. You could try checking the data guides and/or
    dictionaries on
    \url{https://dialysisdata.org/content/dialysis-facility-report-data}
    to find out.}
\end{minipage}
\end{table}

Not all variables used \cite{grieco2017} are included here. Some
variables will need to be transformed to be comparable to what is in
the paper. For example, net investment in stations in year $t$ is the
difference between the number of stations in year $t+1$ and year in
$t$.
\begin{lstlisting}
#' Create lags for panel data.
#'
#' This function creates lags (or leads) of panel data variables.
#' Input data should be sorted by i, t --- e.g.
#' df <- df[order(i,t),]
#' @param x Vector or matrix to get lags of.
#' @param i unit index
#' @param t time index
#' @param lag How many time periods to lag. Can be negative if leading
#' values are desired.
#' @return Lagged copy of x.
panel.lag <- function(x, i, t, lag=1) {
  if (!identical(order(i,t),1:length(i))) {
    stop("inputs not sorted.")
  }
  if (is.matrix(x)) {
    return(apply(x,MARGIN=2,FUN=function(c) { panel.lag(c,i,t,lag) }))
  }
  if (length(i) != length(x) || length(i) != length(t) ) {
    stop("Inputs not same length")
  }
  if (lag>0) {
    x.lag <- x[1:(length(x)-lag)]
    x.lag[i[1:(length(i)-lag)]!=i[(1+lag):length(i)] ] <- NA
    x.lag[t[1:(length(i)-lag)]+lag!=t[(1+lag):length(i)] ] <- NA
    val <- (c(rep(NA,lag),x.lag))
  } else if (lag<0) {
    lag <- abs(lag)
    x.lag <- x[(1+lag):length(x)]
    x.lag[i[1:(length(i)-lag)]!=i[(1+lag):length(i)] ] <- NA
    x.lag[t[1:(length(i)-lag)]+lag!=t[(1+lag):length(i)] ] <- NA
    val <- (c(x.lag,rep(NA,lag)))
  } else { # lag=0
    return (x)
  }
  if (class(x)=="Date" & class(val)=="numeric") {
    stopifnot(0==as.numeric(as.Date("1970-01-01")))
    val <- as.Date(val, origin="1970-01-01")
  }
  return(val)
}
dialysis <- dialysis[order(dialysis$provfs, dialysis$year), ]
dialysis$change.stations <- panel.lag(dialysis$stations, dialysis$provfs,
                                      dialysis$year, lag=-1) - dialysis$stations
\end{lstlisting}  
Net hiring can similarly be created. State inspection rates can be
created as
\begin{lstlisting}
dialysis$inspected.this.year <- (dialysis$days.since.inspection>=0 &
                                 dialysis$days.since.inspection<365)
dialysis$state.inspect.rate <- with(dialysis,
                                    ave(inspected.this.year,state,
                                        year, FUN=function(x) {mean(x,na.rm=TRUE)}))
\end{lstlisting}

\clearpage

\subsection{Descriptive statistics}

Show some summary statistics with
\begin{lstlisting}
summary(dialysis)
\end{lstlisting}
The builtin summary command is easy to use, but it does not quite
provide all the information that we might want. For example, it does
not show the standard deviation of each variable. We can calculate the
standard deviation of a single variable with
\begin{lstlisting}
sd(dialysis$patient.months, na.rm=TRUE)
\end{lstlisting}
or, we could calculate the standard deviation of all variables using
the apply command,
\begin{lstlisting}
apply(dialsys, 2, FUN=function(x) { sd(x, na.rm=TRUE) })
\end{lstlisting}
We can use the apply command to create something similar to part of
Table 1 of \cite{grieco2017}.
\begin{lstlisting}
dialysis$for.profit <- dialysis$profit.status=="For Profit"
var.names <- c("days.since.inspection","for.profit")
tab1 <- t(apply(dialysis[,var.names], 2, 
             function(input) {
               x <- as.numeric(input)
               c(mean(x,na.rm=TRUE), sd(x,na.rm=TRUE), sum(!is.na(x)))
             }))
colnames(tab1) <- c("Mean", "St. Dev.", "N")
tab1
\end{lstlisting}
There are multiple R packages for converting R matrices (like tab1)
into a nicely formatted table. The stargazer package provides latex
and html tables. Install it by typing
\begin{lstlisting}
install.packages("stargazer")
\end{lstlisting}
You only need to install it once, but you need to load it in R session
before you use it. Load it with
\begin{lstlisting}
library(stargazer)
\end{lstlisting}
You could make a latex table with 
\begin{lstlisting}
stargazer(tab1,type="latex")
\end{lstlisting}
There are other packages for other
formats. \href{http://rmarkdown.rstudio.com/lesson-7.html} {Here is a
  list of a few.} 

\begin{problem}
  Create a table or tables containing similar information as Tables 1,
  2, and 3 of \cite{grieco2017}. Not all variables from the paper are
  available here, but include what you can.  You could also choose to
  display additional summary statistics or variables. Comment on any
  large differences from the paper or other anomalies.
  
  I did not make any intentional mistakes while creating the dialysis
  data frame from the raw data in the dialysis facility reports, but I
  am also not certain that there are not errors or omissions. If you
  suspect there is a problem, you can look at the
  downloadDialysisData.R code and the documentation at
  \url{https://dialysisdata.org/content/dialysis-facility-report-data}
  to figure out what might be wrong, and change downloadDialysisData.R
  to fix the problem. 
\end{problem}

\subsection{Descriptive plots}

Let's make some plots of the data. Here's a histogram of patient months
\begin{lstlisting}
hist(dialysis$patient.months)
\end{lstlisting}
Here's a scatter plot of total staff and patient months
\begin{lstlisting}
plot(x=dialysis$total.staff, y=dialysis$patient.months)
\end{lstlisting}
The builtin R plotting commands are convenient, but the ggplot2
package can create nicer looking figures. Creating nicer figures is
not without a cost; the syntax for ggplot2 is far more verbose than
the builtin plotting commands.

The following uses ggplot2 to show the state inspection rates vs year.
\begin{lstlisting}
library(ggplot2) ## install.packages("ggplot2") 
## plot state inspection rates vs year
fig.df <- aggregate(inspected.this.year ~ state*year, data=dialysis, FUN=mean)
fig <- ggplot(data=fig.df, aes(x=year, y=inspected.this.year,
                               colour=state)) +
  geom_point() +
  theme_minimal() 
print(fig)
\end{lstlisting}
The plotly package can convert ggplot2 figures into interactive
webpages. Among other things, it lets you hover over any point in a
scatter plot and see the underlying data. 
\begin{lstlisting}
library(plotly) ## instack.packages("plotly")
ggplotly(fig)
\end{lstlisting}

\begin{problem}
  Create scatter plots of output, labor, capital, and quality. 
  Are there any strange patterns or other obvious problems with the data?
\end{problem}

%%%%%%%%%%%%%%%%%%%%%%%%%%%%%%%%%%%%%%%%%%%%%%%%%%%%%%%%%%%%%%%%%%%%%%%%%%%%%%%%
\subsection{Infection rate and incentive shifters}

The command for regression in R is ``lm'', which stands for linear
model. For fixed effects regression, there is ``plm''. Here's an
example:
\begin{lstlisting}
library(plm)
tab4.col1 <- plm(pct.septic ~ I(days.since.inspection/365) +
                 patient.age + pct.female + patient.esrd.years + pct.fistula +
                 comorbidities + hemoglobin,
                 data=dialysis, index="provfs")) 
summary(tab4.col1)
\end{lstlisting}
\begin{problem}
  Reproduce table 4 of \cite{grieco2017}. Comment on notable
  differences. 
\end{problem}


\section{Production function estimation}

\begin{problem}
  Reproduce columns 2,3, 5, and 6 of table 5. Use the lm() command for
  OLS and plm() for fixed effects. 
\end{problem}

TO BE CONTINUED


\bibliographystyle{jpe}
\bibliography{../../565}

\end{document}
